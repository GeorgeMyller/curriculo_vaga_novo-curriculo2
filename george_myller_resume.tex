\documentclass{article}
\usepackage[utf8]{inputenc}
\usepackage{geometry}
\geometry{a4paper, margin=1in}

\title{GEORGE MYLLER ESTEVES DE SOUZA}
\author{(+351) 912331561 | george.myller@gmail.com | linkedin.com/in/george-m-souza | github.com/GeorgeMyller}

\begin{document}
\maketitle

\section*{Objetivo}
\noindent Atuar como Cientista de Dados

\section*{Experiência Profissional}

\subsection*{Desenvolvedor de Software \textit{(Freelancer) - Janeiro/2024 -- Atual}}
\begin{itemize}
    \item Desenvolvimento de soluções em Python para automação de processos, análise de dados e integração de APIs.
    \item Criação de dashboards interativos com Streamlit e visualizações personalizadas para tomada de decisão.
    \item Projetos envolvendo Machine Learning, LLMs (Large Language Models) e integração com CrewAI e API Gemini.
    \item Desenvolvimento de chatbots inteligentes e automações para mídias sociais, utilizando Flask e bibliotecas de IA.
    \item Utilização de Git e Docker.
    \item Aplicação de conceitos de ETL, automação de relatórios, desenvolvimento de APIs RESTful e manipulação de dados com Pandas e NumPy.
\end{itemize}

\subsection*{Operador de Logística \textit{(Siemens Gamesa) - Maio/2022 -- Maio/2024}}
\begin{itemize}
    \item Otimização de processos logísticos e eficiência operacional em ambiente fabril.
    \item Controle de estoque, movimentação de materiais e gestão de insumos.
    \item Suporte a sistemas integrados de produção.
\end{itemize}

\section*{Habilidades Técnicas}
Python, Streamlit, Flask, Git, Docker, SQL, Pandas, NumPy, Scikit-learn, LLMs, CrewAI, API Gemini, ETL, APIs RESTful, Machine Learning, Data Science, Engenharia de Dados, Inteligência Artificial.

\section*{Formação Acadêmica}

\subsection*{Mestrado em Ciências Veterinárias - Universidade Federal do Paraná UFPR (2020)}
\subsection*{Licenciatura em Ciências Biológicas - Universidade Federal de Minas Gerais UFMG (2015)}

\section*{Cursos de Aperfeiçoamento}
Microcredencial em Fundamentos de Aprendizagem Automática - Universidade de Aveiro (2025), Fundamentos de Data Science e Inteligência Artificial - Data Science Academy (2024), Microcredencial em Programação em Python para análise de dados - Universidade de Aveiro (2024), Fundamentos de Engenharia de Dados - Data Science Academy (2024), Imersão Inteligência Artificial 2ª Edição - Alura (2024), Initial Course on CrewAI - DeepLearning.AI (2024), Fundamentos de Linguagem Python para Análise de Dados e Data Science - Data Science Academy (2024), Agentes Inteligentes - CrewAI - Canal Sandeco (2025), Python para Inteligência Artificial - Canal Sandeco (2025).

\section*{Informações Adicionais}
Podcaster Fundador e Co-Fundador: Tribo Reptiliana e Meu Exótico Podcast (2020 – 2023), Atuação como Professor Universitário – UniCesumar (2020).

\end{document}