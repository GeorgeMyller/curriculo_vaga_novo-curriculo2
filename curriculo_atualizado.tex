\documentclass{article}
\usepackage[utf8]{inputenc}
\usepackage{hyperref}
\usepackage{geometry}
\geometry{a4paper, margin=1in}

\title{GEORGE MYLLER ESTEVES DE SOUZA}
\author{\href{mailto:george.myller@gmail.com}{george.myller@gmail.com} | (+351) 912331561 | \href{https://linkedin.com/in/george-m-souza}{LinkedIn} | \href{https://github.com/GeorgeMyller}{GitHub}}
\date{}

\begin{document}
\maketitle

\section*{Objetivo}
 Atuar como Desenvolvedor Python/Cientista de Dados/Engenheiro de Machine Learning

\section*{Experiência Profissional}

\subsection*{Desenvolvedor de Software \newline Freelancer \newline Janeiro/2024 – Atual}
\begin{itemize}
    \item Desenvolvimento de soluções em Python para automação de processos, análise de dados e integração de APIs (ex: integração com CrewAI e API Gemini).
    \item Criação de dashboards interativos utilizando Streamlit para visualização de dados.
    \item Projetos envolvendo Machine Learning e LLMs (Large Language Models), com foco em [mencionar exemplos específicos de projetos e resultados].
    \item Desenvolvimento de chatbots inteligentes utilizando Flask.
    \item Aplicação de conceitos de ETL (Extract, Transform, Load), automação de relatórios, desenvolvimento de APIs RESTful e manipulação de dados com Pandas e NumPy.
    \item Gerenciamento de código utilizando Git e Docker (nível básico).
\end{itemize}

\subsection*{Operador de Logística \newline Siemens Gamesa Renewable Energy Blades S.A \newline Maio/2022 – Maio/2024}
\begin{itemize}
    \item Otimização de processos logísticos e eficiência operacional.
    \item Controle de estoque, movimentação de materiais, gestão de insumos e suporte a sistemas integrados de produção.
\end{itemize}

\subsection*{Biólogo Responsável \newline Animais Silvestres e Exóticos DinoPet \newline Março/2018 – Janeiro/2022}
\begin{itemize}
    \item Gestão de equipe e planejamento estratégico.
    \item Elaboração de relatórios técnicos e científicos.
    \item Controle de estoque, atendimento ao cliente, vendas e marketing digital.
\end{itemize}

\section*{Formação Acadêmica}

\subsection*{Mestrado em Ciências Veterinárias \newline Universidade Federal do Paraná UFPR \newline 2020}

\subsection*{Licenciatura em Ciências Biológicas \newline Universidade Federal de Minas Gerais UFMG \newline 2015}

\section*{Cursos de Aperfeiçoamento Profissional}

\begin{itemize}
    \item Microcredencial em Fundamentos de Aprendizagem Automática - Universidade de Aveiro (2025)
    \item Fundamentos de Data Science e Inteligência Artificial - Data Science Academy (2024)
    \item Microcredencial em Programação em Python para análise de dados - Universidade de Aveiro (2024)
    \item Fundamentos de Engenharia de Dados - Data Science Academy (2024)
    \item Imersão Inteligência Artificial 2ª Edição - Alura (2024)
    \item Initial Course on CrewAI - DeepLearning.AI (2024)
    \item Fundamentos de Linguagem Python para Análise de Dados e Data Science - Data Science Academy (2024)
    \item Agentes Inteligentes - CrewAI - Canal Sandeco (2025)
    \item Python para Inteligência Artificial - Canal Sandeco (2025)
\end{itemize}

\section*{Habilidades}

\subsection*{Idiomas}
Inglês (Intermediário - B2)

\subsection*{Habilidades Técnicas}
Python, Pandas, NumPy, Scikit-learn, Streamlit, Flask, Git, Docker, SQL, Machine Learning, LLMs, CrewAI, API Gemini, ETL, desenvolvimento de APIs RESTful

\subsection*{Habilidades Comportamentais}
Liderança, organização, comunicação, proatividade, resolução de problemas, trabalho em equipe

\section*{Informações Adicionais}

\begin{itemize}
    \item Podcaster Fundador e Co-Fundador: Tribo Reptiliana e Meu Exótico Podcast (2020 – 2023)
    \item Atuação como Professor Universitário – UniCesumar (2020)
\end{itemize}

\end{document}
