\documentclass[a4paper,10pt]{article} % Changed font size to 10pt for better fit potentially
\usepackage[utf8]{inputenc}
\usepackage[T1]{fontenc} % Added for font encoding
\usepackage{helvet} % Added for Helvetica-like sans-serif font
\renewcommand{\familydefault}{\sfdefault} % Set sans-serif as default
\usepackage[portuguese]{babel}
\usepackage{geometry}
\usepackage{hyperref}
\usepackage{titlesec} % Added for section title customization
\usepackage{sectsty} % Added for section styling (alternative/complement)
\usepackage{enumitem}

\geometry{a4paper, margin=1in}

% Customize section titles
\titleformat{\section}
  {\normalfont\Large\bfseries\MakeUppercase} % Format: Normal, Large, Bold, Uppercase
  {}
  {0em}
  {} % No label
  [\titlerule] % Rule after title
\titlespacing*{\section}{0pt}{3.5ex plus 1ex minus .2ex}{2.3ex plus .2ex} % Spacing

% Customize subsection titles (e.g., Job Titles) - Keep bold as per image
\titleformat*{\subsection}{\large\bfseries}
\titlespacing*{\subsection}{0pt}{2ex plus .5ex minus .2ex}{1ex plus .2ex}

% Remove default paragraph indentation
\setlength{\parindent}{0pt}
% Add space between paragraphs
\setlength{\parskip}{0.5em}
\setlist{itemsep=0pt, topsep=0pt, partopsep=0pt, parsep=0pt}

\begin{document}

% Remove default title block
% \title{Currículo - George Myller Esteves de Souza}
% \author{}
% \date{}
% \maketitle

% Custom Header
\begin{center}
    {\fontsize{16}{18}\bfseries\MakeUppercase{George Myller Esteves de Souza}}\\ % Corrected line break
    \vspace{0.5em} % Space after name
    Endereço: Forca - Aveiro \\ % Corrected line break
    Contatos: (+351) 912331561 | george.myller@gmail.com \\ % Corrected line break
    LinkedIn: \url{linkedin.com/in/george-m-souza} \\ % Corrected line break
    GitHub: \url{github.com/GeorgeMyller}
\end{center}
\vspace{1em} % Space after header

% Keep section content, but remove the old \section*{Informações Pessoais} block
% \section*{Informações Pessoais}
% \begin{itemize}
%     \item Endereço: Fora - Aveiro
%     \item Contatos: (+351) 912331561 | george.myller@gmail.com
%     \item LinkedIn: \url{linkedin.com/in/george-m-souza}
%     \item GitHub: \url{github.com/GeorgeMyller}
% \end{itemize}



\section*{Resumo Profissional}
Profissional com Mestrado em Ciências Veterinárias (UFPR) e Licenciatura em Ciências Biológicas (UFMG), em transição estratégica para a área de Tecnologia da Informação, com foco em Desenvolvimento de Software, Análise de Dados e Inteligência Artificial. Atuação atual como Desenvolvedor Freelancer, com experiência prática no desenvolvimento de soluções em Python para automação de processos, integração de APIs, análise de dados e machine learning. Conduzi projetos com LLMs, CrewAI e API Gemini, além da criação de dashboards interativos, chatbots inteligentes e aplicações com Streamlit e Flask. Domino ferramentas como Git, Docker (básico) e SQL (básico), além de bibliotecas como Pandas, NumPy e Scikit-learn. Possuo ampla qualificação técnica em Data Science, Engenharia de Dados, Inteligência Artificial e Python para análise de dados, por instituições como Universidade de Aveiro, Data Science Academy, Alura e DeepLearning.AI. Trago uma bagagem consistente em gestão, liderança e planejamento estratégico, com vivência como Biólogo Responsável e Professor Universitário. Tenho perfil analítico, autônomo e orientado a resultados. Inglês nível B2.

\section*{Formação Acadêmica}
\begin{itemize}
    \item Mestrado em Ciências Veterinárias - Universidade Federal do Paraná UFPR (2020) % Removed bold, section title handles emphasis
    \item Licenciatura em Ciências Biológicas - Universidade Federal de Minas Gerais UFMG (2015) % Removed bold
\end{itemize}

\section*{Experiência Profissional}
\subsection*{Freelancer} % Subsection for company/context
\textbf{Desenvolvedor de Software} \hfill Janeiro/2024 – Atual % Job title bold, date aligned
\begin{itemize}
    \item Desenvolvimento de soluções em Python para automação de processos, análise de dados e integração de APIs.
    \item Criação de dashboards interativos com Streamlit e visualizações personalizadas para tomada de decisão baseada em dados.
    \item Projetos envolvendo Machine Learning, LLMs (Large Language Models) e integração com ferramentas como CrewAI e API Gemini.
    \item Desenvolvimento de chatbots inteligentes e automações para mídias sociais, utilizando Flask e bibliotecas de IA.
    \item Utilização de versionamento com Git e containers básicos com Docker.
    \item Aplicação de conceitos de ETL, automação de relatórios, desenvolvimento de APIs RESTful e manipulação de dados com Pandas e NumPy.
\end{itemize}

\subsection*{Siemens Gamesa Rewable Energy Blades S.A}
\textbf{Operador de Logística} \hfill Maio/2022 – Maio/2024
\begin{itemize}
    \item Atuação em ambiente fabril com foco na otimização de processos logísticos e eficiência operacional.
    \item Experiência com controle de estoque, movimentação de materiais, gestão de insumos e suporte a sistemas integrados de produção.
\end{itemize}

\subsection*{Animais Silvestres e Exóticos DinoPet}
\textbf{Biólogo Responsável} \hfill Março/2018 – Janeiro/2022
\begin{itemize}
    \item Gestão de equipe (contratação, treinamento e acompanhamento de estagiários e bolsistas).
    \item Elaboração de relatórios técnicos e científicos e condução de pesquisas analíticas para aumento de eficiência reprodutiva.
    \item Controle de estoque, atendimento ao cliente, vendas e responsável pelo marketing digital da empresa.
    \item Planejamento estratégico e tomada de decisão baseada em análise de indicadores.
\end{itemize}

\section*{Cursos de Aperfeiçoamento Profissional}
\begin{itemize}
    \item Microcredencial em Fundamentos de Aprendizagem Automática - Universidade de Aveiro (2025)
    \item Fundamentos de Data Science e Inteligência Artificial - Data Science Academy (2024)
    \item Microcredencial em Programação em Python para análise de dados - Universidade de Aveiro (2024)
    \item Fundamentos de Engenharia de Dados - Data Science Academy (2024)
    \item Imersão Inteligência Artificial 2ª Edição - Alura (2024)
    \item Initial Course on CrewAI - DeepLearning.AI (2024)
    \item Fundamentos de Linguagem Python para Análise de Dados e Data Science - Data Science Academy (2024)
    \item Agentes Inteligentes - CrewAI - Canal Sandeco (2025)
    \item Python para Inteligência Artificial - Canal Sandeco (2025)
\end{itemize}

\section*{Idiomas}
\begin{itemize}
    \item Inglês: Intermediário - B2
\end{itemize}

\section*{Informações Adicionais}
\begin{itemize}
    \item Podcaster Fundador e Co-Fundador: Tribo Reptiliana e Meu Exótico Podcast (2020 – 2023)
    \item Atuação como Professor Universitário – UniCesumar (2020)
\end{itemize}

\end{document}
