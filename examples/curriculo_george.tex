\documentclass{article}
\usepackage[utf8]{inputenc}
\usepackage{hyperref}
\usepackage{fontawesome}

\title{George Myller Esteves de Souza}
\author{\href{mailto:george.myller@gmail.com}{george.myller@gmail.com} | (+351) 912331561 | \href{https://linkedin.com/in/george-m-souza}{LinkedIn} | \href{https://github.com/GeorgeMyller}{GitHub}}
\date{}

\begin{document}
\maketitle

\section*{Objetivo}
 Atuar como Data Analyst

\section*{Resumo}
Desenvolvedor de Software com experiência em análise de dados, machine learning e desenvolvimento de aplicações em Python. Procuro uma posição como Data Analyst para aplicar minhas habilidades em projetos desafiadores e contribuir para o sucesso da equipe.

\section*{Experiência Profissional}
\subsection*{Desenvolvedor de Software \textit{(Freelancer) - Janeiro/2024 - Atual}}
* Desenvolvimento de soluções em Python para automação de processos, análise de dados e integração de APIs.
* Criação de dashboards interativos com Streamlit e visualizações personalizadas para tomada de decisão.
* Projetos envolvendo Machine Learning, LLMs (Large Language Models) e integração com CrewAI e API Gemini.
* Desenvolvimento de chatbots inteligentes e automações para mídias sociais, utilizando Flask e bibliotecas de IA.
* Utilização de Git e Docker.
* Aplicação de conceitos de ETL, automação de relatórios, desenvolvimento de APIs RESTful e manipulação de dados com Pandas e NumPy.

\subsection*{Operador de Logística \textit{(Siemens Gamesa) - Maio/2022 - Maio/2024}}
* Otimização de processos logísticos e eficiência operacional em ambiente fabril.
* Controle de estoque, movimentação de materiais e gestão de insumos.
* Suporte a sistemas integrados de produção.

\subsection*{Biólogo Responsável \textit{(Animais Silvestres e Exóticos DinoPet) - Março/2018 - Janeiro/2022}}
* Gestão de equipe (contratação, treinamento e acompanhamento).
* Elaboração de relatórios técnicos e científicos.
* Condução de pesquisas analíticas para aumento de eficiência reprodutiva.
* Controle de estoque, atendimento ao cliente, vendas e marketing digital.
* Planejamento estratégico e tomada de decisão baseada em análise de indicadores.

\section*{Formação Acadêmica}
\subsection*{Mestrado em Ciências Veterinárias \textit{(UFPR) - 2020}}
\subsection*{Licenciatura em Ciências Biológicas \textit{(UFMG) - 2015}}

\section*{Cursos}
* Microcredencial em Fundamentos de Aprendizagem Automática - Universidade de Aveiro (2025)
* Fundamentos de Data Science e Inteligência Artificial - Data Science Academy (2024)
* Microcredencial em Programação em Python para análise de dados - Universidade de Aveiro (2024)
* Fundamentos de Engenharia de Dados - Data Science Academy (2024)
* Imersão Inteligência Artificial 2ª Edição - Alura (2024)
* Initial Course on CrewAI - DeepLearning.AI (2024)
* Fundamentos de Linguagem Python para Análise de Dados e Data Science - Data Science Academy (2024)
* Agentes Inteligentes - CrewAI - Canal Sandeco (2025)
* Python para Inteligência Artificial - Canal Sandeco (2025)

\section*{Habilidades}
* Python
* SQL
* Machine Learning
* Análise de Dados
* Visualização de Dados (Streamlit)
* Pandas
* NumPy
* APIs RESTful
* Git
* Docker
* ETL
* Inglês (Intermediário - B2)

\end{document}