\documentclass{article}
\usepackage[utf8]{inputenc}
\usepackage{hyperref}

\title{Currículo - George Myller Esteves de Souza}
\author{\href{mailto:george.myller@gmail.com}{george.myller@gmail.com} | (+351) 912331561 | \href{https://linkedin.com/in/george-m-souza}{LinkedIn} | \href{https://github.com/GeorgeMyller}{GitHub}}
\date{}

\begin{document}
\maketitle

\section*{Objetivo}
\textbf{Atuar como Analista de Dados}

\section*{Resumo}
Analista de Dados com experiência em desenvolvimento de software, análise de dados e Machine Learning, buscando uma posição desafiadora que utilize minhas habilidades em Python, SQL e visualização de dados para contribuir com o sucesso da organização.

\section*{Experiência Profissional}

\subsection*{Desenvolvedor de Software \textit{(Freelancer) - Janeiro/2024 – Atual}}
\begin{itemize}
    \item Desenvolvimento de soluções em Python para automação de processos, análise de dados e integração de APIs.
    \item Criação de dashboards interativos com Streamlit e visualizações personalizadas para tomada de decisão baseada em dados.
    \item Projetos envolvendo Machine Learning, LLMs (Large Language Models) e integração com ferramentas como CrewAI e API Gemini.
    \item Desenvolvimento de chatbots inteligentes e automações para mídias sociais, utilizando Flask e bibliotecas de IA.
    \item Utilização de versionamento com Git e containers básicos com Docker.
    \item Aplicação de conceitos de ETL, automação de relatórios, desenvolvimento de APIs RESTful e manipulação de dados com Pandas e NumPy.
\end{itemize}

\section*{Formação Acadêmica}

\subsection*{Mestrado em Ciências Veterinárias - Universidade Federal do Paraná UFPR (2020)}
\subsection*{Licenciatura em Ciências Biológicas - Universidade Federal de Minas Gerais UFMG (2015)}

\section*{Cursos}
\begin{itemize}
    \item Microcredencial em Fundamentos de Aprendizagem Automática - Universidade de Aveiro (2025)
    \item Fundamentos de Data Science e Inteligência Artificial - Data Science Academy (2024)
    \item Microcredencial em Programação em Python para análise de dados - Universidade de Aveiro (2024)
    \item Fundamentos de Engenharia de Dados - Data Science Academy (2024)
    \item Imersão Inteligência Artificial 2ª Edição - Alura (2024)
    \item Initial Course on CrewAI - DeepLearning.AI (2024)
    \item Fundamentos de Linguagem Python para Análise de Dados e Data Science - Data Science Academy (2024)
    \item Agentes Inteligentes - CrewAI - Canal Sandeco (2025)
    \item Python para Inteligência Artificial - Canal Sandeco (2025)
\end{itemize}

\section*{Habilidades}
\textbf{Linguagens de Programação:} Python, SQL \newline
\textbf{Ferramentas e Tecnologias:} Pandas, NumPy, Streamlit, Docker, Git, APIs RESTful, Machine Learning, LLMs, ETL, CrewAI, API Gemini \newline
\textbf{Idiomas:} Inglês (Intermediário - B2)

\end{document}