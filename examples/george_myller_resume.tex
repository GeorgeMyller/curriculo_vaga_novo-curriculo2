\documentclass{article}
\usepackage[utf8]{inputenc}
\usepackage{geometry}
\geometry{a4paper, margin=1in}

\title{George Myller Esteves de Souza}
\author{george.myller@gmail.com | +351 912331561 | linkedin.com/in/george-m-souza | github.com/GeorgeMyller}

\begin{document}
\maketitle

\section*{Objetivo}
\textbf{Atuar como Analista de Dados}

\section*{Resumo Profissional}
Profissional com Mestrado em Ciências Veterinárias (UFPR) e Licenciatura em Ciências Biológicas (UFMG), em transição estratégica para a área de Tecnologia da Informação, com foco em Desenvolvimento de Software, Análise de Dados e Inteligência Artificial. Experiência prática em desenvolvimento de soluções em Python para automação de processos, integração de APIs, análise de dados e machine learning.  Domínio de ferramentas como Pandas, NumPy, Scikit-learn e experiência na criação de dashboards interativos com Streamlit, demonstrando forte capacidade de visualização e comunicação de dados. Experiência com LLMs e integração com APIs como CrewAI e Gemini. Experiência em projetos com ETL, automação de relatórios e manipulação de dados.  Proficiência em Git e familiaridade com Docker e SQL. Perfil analítico, autônomo e orientado a resultados. Inglês nível B2.

\section*{Experiência Profissional}
\subsection*{Desenvolvedor de Software \textit{(Freelancer, Janeiro/2024 – Atual)}}
Desenvolvimento de soluções em Python para automação de processos, análise de dados e integração de APIs. Criação de dashboards interativos com Streamlit e visualizações personalizadas para tomada de decisão baseada em dados. Projetos envolvendo Machine Learning e integração com ferramentas como CrewAI e API Gemini. Desenvolvimento de aplicações com Flask.  Utilização de versionamento com Git. Aplicação de conceitos de ETL e automação de relatórios. Manipulação de dados com Pandas e NumPy.
\subsection*{Operador de Logística \textit{(Siemens Gamesa Rewable Energy Blades S.A, Maio/2022 – Maio/2024)}}
Otimização de processos logísticos, controle de estoque, movimentação de materiais e suporte a sistemas integrados de produção.
\subsection*{Biólogo Responsável \textit{(Animais Silvestres e Exóticos DinoPet, Março/2018 – Janeiro/2022)}}
Gestão de equipe, elaboração de relatórios técnicos, condução de pesquisas analíticas, controle de estoque, atendimento ao cliente, vendas e marketing digital. Planejamento estratégico e tomada de decisão baseada em análise de indicadores.

\section*{Formação Acadêmica}
\subsection*{Mestrado em Ciências Veterinárias \textit{(Universidade Federal do Paraná (UFPR), 2020)}}
\subsection*{Licenciatura em Ciências Biológicas \textit{(Universidade Federal de Minas Gerais (UFMG), 2015)}}

\section*{Habilidades}
Python, Pandas, NumPy, Scikit-learn, Machine Learning, Streamlit, Flask, Git, SQL (básico), Docker (básico), Data Visualization, ETL, Análise de Dados,  Inglês (B2)

\section*{Cursos}
* Microcredencial em Fundamentos de Aprendizagem Automática (Universidade de Aveiro, 2025)
* Fundamentos de Data Science e Inteligência Artificial (Data Science Academy, 2024)
* Microcredencial em Programação em Python para análise de dados (Universidade de Aveiro, 2024)
* Fundamentos de Engenharia de Dados (Data Science Academy, 2024)
* Imersão Inteligência Artificial 2ª Edição (Alura, 2024)
* Initial Course on CrewAI (DeepLearning.AI, 2024)

\section*{Informações Adicionais}
Podcaster (Fundador e Co-Fundador: Tribo Reptiliana e Meu Exótico Podcast, 2020-2023). Professor Universitário (UniCesumar, 2020).

\end{document}